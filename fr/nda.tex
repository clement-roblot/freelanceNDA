\documentclass[]{karlito}
%\documentclass[remplissage, impression]{karlito}

\usepackage{karlito}

\usepackage{textpos}
\usepackage{digsig}



\title{Accord de confidentiallité}
\author{Clément Roblot}


\usepackage[calc,useregional=numeric]{datetime2}
\newcount\myct
\newcount\datecount
\newcommand{\todayPlus}[1]{%
    \DTMsavenow{mydate}
    \DTMsaveddateoffsettojulianday{mydate}{#1}{\myct}
    \DTMsavejulianday{mydate}{\number\myct}
    \DTMusedate{mydate}
}  



%\conditionnalForm{text}{nameFormFieldIfTextEmpty}
\newcommand{\conditionnalForm}[2]{\ifthenelse{\equal{#1}{}}{\TextField[#2]{}}{#1}}





%%%%%%%%%%%%%%%%% Project
\newcommand{\projectName}{Refactoring d'un logiciel de chronométrage}
\newcommand{\startDateProject}{\todayPlus{0}}
\newcommand{\stopDateProject}{\todayPlus{364}}


%%%%%%%%%%%%%%%%% Beneficiary 
\newcommand{\nameBeneficiaryCompanie}{MONSIEUR CLEMENT ROBLOT}
\newcommand{\nameBeneficiaryPerson}{Clément Roblot}
\newcommand{\titleBeneficiaryPerson}{}
\newcommand{\addressBeneficiary}{51 avenue de la libération 94100 Saint Maur des Fossés (France)}
\newcommand{\locationBeneficiary}{Chiang Mai (Thailand)}
\newcommand{\siretBeneficiary}{831 951 306 00016}

%%%%%%%%%%%%%%%%% Issuing
\newcommand{\nameIssuingCompanie}{SARL A.T.S.}
\newcommand{\nameIssuingPerson}{}
\newcommand{\titleIssuingPerson}{}
\newcommand{\locationIssuing}{}
\newcommand{\dateIssuing}{}
\newcommand{\addressIssuing}{20 Avenue du Général Grollier 34570 Pignan (France)}
\newcommand{\siretIssuing}{443 182 233 00033}



\begin{document}

\begin{center}
\textbf{ACCORD DE CONFIDENTIALITÉ}
\end{center}

~\newline
~\newline

\textbf{ENTRE LES SOUSSIGNÉ(E)S :}
~\newline

\textbf{\nameIssuingCompanie}\\
Dont le siège est situé au \addressIssuing\\
N° de SIRET : \siretIssuing\\
Représentée par \conditionnalForm{\nameIssuingPerson}{name=nameIssuingPerson,width=12cm}\\
Ci-après désignée par « \nameIssuingCompanie »

\begin{flushright}\textbf{D’une part,}\end{flushright}

\textbf{\nameBeneficiaryCompanie}\\
Dont le siège est situé au \addressBeneficiary\\
N° de SIRET : \siretBeneficiary\\
Représentée par \nameBeneficiaryPerson,\\
Ci-après désignée par « \nameBeneficiaryCompanie »

\begin{flushright}\textbf{D’autre part,}\end{flushright}


\nameIssuingCompanie\,et \nameBeneficiaryCompanie\,seront ci-après individuellement ou collectivement désignés par la « Partie » ou les « Parties ».

~\newline
~\newline
~\newline
\textbf{PRÉAMBULE :}
~\newline

Dans le cadre d’une consultation en vue d’étudier l’opportunité de la réalisation du projet d’étude intitulé : \projectName, les parties vont se communiquer des informations relatives à ce projet d’étude.

La communication d’informations est réalisée dans les conditions définies ci-après.



\newpage
\chapter{DÉFINITIONS}
\section{}La « Partie Émettrice » désigne la partie qui communique ses informations confidentielles à l’autre partie.

\section{}La « Partie Bénéficiaire » désigne la partie qui reçoit les informations confidentielles de l’autre partie.

\section{}Les Parties vont s’échanger des documents, données, échantillons, savoir faire, prototypes, informations, études et outils relatifs à \projectName, ci-après désignés globalement « les informations ».




\chapter{CONFIDENTIALITÉ}
\section{}La Partie Bénéficiaire s’engage à garder strictement confidentiel et à ne pas divulguer ou communiquer à des tiers, par quelque moyen que ce soit, les informations qui lui seront transmises par la Partie Emettrice ou auxquelles elle aura accès à l’occasion de l’exécution du présent accord.

\section{}La Partie Bénéficiaire prendra toutes les mesures nécessaires pour préserver le caractère confidentiel des informations. Ces mesures ne pourront pas être inférieures à celles prises par elle pour la protection de ses propres informations confidentielles.


\section{}\label{subContract}La Partie Bénéficiaire s’engage à ne communiquer lesdites informations qu’aux membres de son personnel appelés à en prendre connaissance et à les utiliser.

Toutefois, la Partie Bénéficiaire pourra communiquer les informations à ses sous-traitants qui pourraient avoir à participer au projet sus mentionné après accord préalable, écrit et exprès de la Partie Émettrice.

\section{}La Partie Bénéficiaire s’engage à prendre toutes les dispositions pour que ses employés et soustraitants, selon l’article \ref{subContract} du présent accord, traitent lesdites informations conformément aux dispositions de confidentialité et d’utilisation du présent accord.







\chapter{UTILISATION DES INFORMATIONS}
\section{}Les informations obtenues par la Partie Bénéficiaire ne pourront être utilisées que pour l’exécution de l’objet du présent accord, visé au préambule. Toute autre utilisation sera soumise à l’autorisation préalable et écrite de la Partie Émettrice.


\section{}En aucun cas, la Partie Bénéficiaire ne pourra se prévaloir sur la base desdites informations d’une quelconque concession de licence ou d’un quelconque droit d’auteur ou de possession antérieure selon la définition du Code de la Propriété Intellectuelle.



\chapter{EXCEPTIONS}
Toutefois, les dispositions prévues au présent accord ne s’appliqueront pas aux informations pour lesquelles la Partie Bénéficiaire pourra prouver :

\begin{itemize}
 \item qu’elle les possédait avant la date de communication par la Partie Emettrice, ou
 \item que ces informations étaient du domaine public avant la date de communication par la Partie Émettrice ou qu’elles y sont entrées par la suite sans qu’une faute puisse être imputée à la Partie Bénéficiaire, ou
 \item qu’elle les a reçues sans obligation de secret d’un tiers autorisé à les divulguer.
\end{itemize}




\chapter{DURÉE}
\section{}Le présent accord prend effet le \startDateProject\,et demeure en vigueur jusqu’à \stopDateProject.

\section{}Les dispositions de confidentialité prévues au présent accord s’appliqueront pendant toute la durée de celui-ci et pendant cinq (5) ans après son échéance ou sa résiliation quelle qu’en soit la cause.




\chapter{LOI APPLICABLE}
Le présent accord est régi par la loi française.


\chapter{LITIGES}
En cas de difficultés sur l'interprétation ou l'exécution du présent contrat, les parties s'efforceront de résoudre leur différend à l'amiable.

En cas de désaccord persistant, les différends seront portés devant le tribunal compétent.

Fait en deux (2) exemplaires originaux, dont un (1) pour chaque partie.

~\newline
~\newline

\begin{minipage}{0.45\textwidth}
À : \locationBeneficiary

Le : \today
\newline

Pour \nameBeneficiaryCompanie\\
Nom : \nameBeneficiaryPerson

Titre : Gérant


Signature :
\newline

\begin{textblock}{0}(1,0)
\includegraphics[width=3cm]{images/signature.png}
\end{textblock}

\end{minipage}%
\hfill
\begin{minipage}{0.45\textwidth}
\begin{Form}
	À : \conditionnalForm{\locationIssuing}{name=locationIssuing,width=5cm}
	
	Le : \conditionnalForm{\dateIssuing}{name=dateIssuing,width=5cm}

	Pour \nameIssuingCompanie
	
	Nom : \conditionnalForm{\nameIssuingPerson}{name=nameIssuingPerson,width=5cm}

	Titre : \conditionnalForm{\titleIssuingPerson}{name=titleIssuingPerson,width=5cm}


	Signature :
	\begin{textblock}{0}(1,0)
	\digsigfield{5cm}{3cm}{My signature}
	\end{textblock}
	
\end{Form}
\end{minipage}%







\appendix



\end{document}
